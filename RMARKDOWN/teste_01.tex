\documentclass[]{article}
\usepackage{lmodern}
\usepackage{amssymb,amsmath}
\usepackage{ifxetex,ifluatex}
\usepackage{fixltx2e} % provides \textsubscript
\ifnum 0\ifxetex 1\fi\ifluatex 1\fi=0 % if pdftex
  \usepackage[T1]{fontenc}
  \usepackage[utf8]{inputenc}
\else % if luatex or xelatex
  \ifxetex
    \usepackage{mathspec}
  \else
    \usepackage{fontspec}
  \fi
  \defaultfontfeatures{Ligatures=TeX,Scale=MatchLowercase}
\fi
% use upquote if available, for straight quotes in verbatim environments
\IfFileExists{upquote.sty}{\usepackage{upquote}}{}
% use microtype if available
\IfFileExists{microtype.sty}{%
\usepackage{microtype}
\UseMicrotypeSet[protrusion]{basicmath} % disable protrusion for tt fonts
}{}
\usepackage[margin=1in]{geometry}
\usepackage{hyperref}
\hypersetup{unicode=true,
            pdftitle={Mineração de texto em pedidos de Lei de Acesso à informação - LAI},
            pdfborder={0 0 0},
            breaklinks=true}
\urlstyle{same}  % don't use monospace font for urls
\usepackage{color}
\usepackage{fancyvrb}
\newcommand{\VerbBar}{|}
\newcommand{\VERB}{\Verb[commandchars=\\\{\}]}
\DefineVerbatimEnvironment{Highlighting}{Verbatim}{commandchars=\\\{\}}
% Add ',fontsize=\small' for more characters per line
\usepackage{framed}
\definecolor{shadecolor}{RGB}{248,248,248}
\newenvironment{Shaded}{\begin{snugshade}}{\end{snugshade}}
\newcommand{\KeywordTok}[1]{\textcolor[rgb]{0.13,0.29,0.53}{\textbf{#1}}}
\newcommand{\DataTypeTok}[1]{\textcolor[rgb]{0.13,0.29,0.53}{#1}}
\newcommand{\DecValTok}[1]{\textcolor[rgb]{0.00,0.00,0.81}{#1}}
\newcommand{\BaseNTok}[1]{\textcolor[rgb]{0.00,0.00,0.81}{#1}}
\newcommand{\FloatTok}[1]{\textcolor[rgb]{0.00,0.00,0.81}{#1}}
\newcommand{\ConstantTok}[1]{\textcolor[rgb]{0.00,0.00,0.00}{#1}}
\newcommand{\CharTok}[1]{\textcolor[rgb]{0.31,0.60,0.02}{#1}}
\newcommand{\SpecialCharTok}[1]{\textcolor[rgb]{0.00,0.00,0.00}{#1}}
\newcommand{\StringTok}[1]{\textcolor[rgb]{0.31,0.60,0.02}{#1}}
\newcommand{\VerbatimStringTok}[1]{\textcolor[rgb]{0.31,0.60,0.02}{#1}}
\newcommand{\SpecialStringTok}[1]{\textcolor[rgb]{0.31,0.60,0.02}{#1}}
\newcommand{\ImportTok}[1]{#1}
\newcommand{\CommentTok}[1]{\textcolor[rgb]{0.56,0.35,0.01}{\textit{#1}}}
\newcommand{\DocumentationTok}[1]{\textcolor[rgb]{0.56,0.35,0.01}{\textbf{\textit{#1}}}}
\newcommand{\AnnotationTok}[1]{\textcolor[rgb]{0.56,0.35,0.01}{\textbf{\textit{#1}}}}
\newcommand{\CommentVarTok}[1]{\textcolor[rgb]{0.56,0.35,0.01}{\textbf{\textit{#1}}}}
\newcommand{\OtherTok}[1]{\textcolor[rgb]{0.56,0.35,0.01}{#1}}
\newcommand{\FunctionTok}[1]{\textcolor[rgb]{0.00,0.00,0.00}{#1}}
\newcommand{\VariableTok}[1]{\textcolor[rgb]{0.00,0.00,0.00}{#1}}
\newcommand{\ControlFlowTok}[1]{\textcolor[rgb]{0.13,0.29,0.53}{\textbf{#1}}}
\newcommand{\OperatorTok}[1]{\textcolor[rgb]{0.81,0.36,0.00}{\textbf{#1}}}
\newcommand{\BuiltInTok}[1]{#1}
\newcommand{\ExtensionTok}[1]{#1}
\newcommand{\PreprocessorTok}[1]{\textcolor[rgb]{0.56,0.35,0.01}{\textit{#1}}}
\newcommand{\AttributeTok}[1]{\textcolor[rgb]{0.77,0.63,0.00}{#1}}
\newcommand{\RegionMarkerTok}[1]{#1}
\newcommand{\InformationTok}[1]{\textcolor[rgb]{0.56,0.35,0.01}{\textbf{\textit{#1}}}}
\newcommand{\WarningTok}[1]{\textcolor[rgb]{0.56,0.35,0.01}{\textbf{\textit{#1}}}}
\newcommand{\AlertTok}[1]{\textcolor[rgb]{0.94,0.16,0.16}{#1}}
\newcommand{\ErrorTok}[1]{\textcolor[rgb]{0.64,0.00,0.00}{\textbf{#1}}}
\newcommand{\NormalTok}[1]{#1}
\usepackage{graphicx,grffile}
\makeatletter
\def\maxwidth{\ifdim\Gin@nat@width>\linewidth\linewidth\else\Gin@nat@width\fi}
\def\maxheight{\ifdim\Gin@nat@height>\textheight\textheight\else\Gin@nat@height\fi}
\makeatother
% Scale images if necessary, so that they will not overflow the page
% margins by default, and it is still possible to overwrite the defaults
% using explicit options in \includegraphics[width, height, ...]{}
\setkeys{Gin}{width=\maxwidth,height=\maxheight,keepaspectratio}
\IfFileExists{parskip.sty}{%
\usepackage{parskip}
}{% else
\setlength{\parindent}{0pt}
\setlength{\parskip}{6pt plus 2pt minus 1pt}
}
\setlength{\emergencystretch}{3em}  % prevent overfull lines
\providecommand{\tightlist}{%
  \setlength{\itemsep}{0pt}\setlength{\parskip}{0pt}}
\setcounter{secnumdepth}{0}
% Redefines (sub)paragraphs to behave more like sections
\ifx\paragraph\undefined\else
\let\oldparagraph\paragraph
\renewcommand{\paragraph}[1]{\oldparagraph{#1}\mbox{}}
\fi
\ifx\subparagraph\undefined\else
\let\oldsubparagraph\subparagraph
\renewcommand{\subparagraph}[1]{\oldsubparagraph{#1}\mbox{}}
\fi

%%% Use protect on footnotes to avoid problems with footnotes in titles
\let\rmarkdownfootnote\footnote%
\def\footnote{\protect\rmarkdownfootnote}

%%% Change title format to be more compact
\usepackage{titling}

% Create subtitle command for use in maketitle
\newcommand{\subtitle}[1]{
  \posttitle{
    \begin{center}\large#1\end{center}
    }
}

\setlength{\droptitle}{-2em}

  \title{Mineração de texto em pedidos de Lei de Acesso à informação - LAI}
    \pretitle{\vspace{\droptitle}\centering\huge}
  \posttitle{\par}
    \author{}
    \preauthor{}\postauthor{}
    \date{}
    \predate{}\postdate{}
  
\usepackage{booktabs}
\usepackage{longtable}
\usepackage{array}
\usepackage{multirow}
\usepackage[table]{xcolor}
\usepackage{wrapfig}
\usepackage{float}
\usepackage{colortbl}
\usepackage{pdflscape}
\usepackage{tabu}
\usepackage{threeparttable}
\usepackage{threeparttablex}
\usepackage[normalem]{ulem}
\usepackage{makecell}

\begin{document}
\maketitle

\subsection{Packages for this routine}\label{packages-for-this-routine}

\section{BASE DE DADOS}\label{base-de-dados}

\subsection{Importação dos dados}\label{importacao-dos-dados}

Caminho do projeto

\begin{Shaded}
\begin{Highlighting}[]
\NormalTok{PATH = }\StringTok{"..;/proj_eSIC_v10/textmining_pt/DATA/"}
\end{Highlighting}
\end{Shaded}

\begin{itemize}
\tightlist
\item
  Pedidos e-SIC
\end{itemize}

\begin{Shaded}
\begin{Highlighting}[]
\NormalTok{FILE =}\StringTok{ "DATA/relatorio_pedidos.ods"}
\NormalTok{db_raw =}\StringTok{ }\NormalTok{readODS}\OperatorTok{::}\KeywordTok{read.ods}\NormalTok{(}\DataTypeTok{file =} \KeywordTok{paste0}\NormalTok{(PATH,FILE), }\DataTypeTok{sheet =} \DecValTok{1}\NormalTok{); }\CommentTok{# dim(db_raw)}
\NormalTok{dbnames =}\StringTok{ }\NormalTok{db_raw[}\DecValTok{1}\NormalTok{,]; db_raw =}\StringTok{ }\NormalTok{db_raw[}\OperatorTok{-}\DecValTok{1}\NormalTok{,]; }
\KeywordTok{colnames}\NormalTok{(db_raw) =}\StringTok{ }\KeywordTok{c}\NormalTok{(}\StringTok{"ProtocoloPedido"}\NormalTok{, }\StringTok{"DataRegistro"}\NormalTok{, }\StringTok{"DATA_PRAZOATEND"}\NormalTok{, }\StringTok{"DESCRI_PEDIDO"}\NormalTok{,}
                     \StringTok{"RESUMO_PEDIDO"}\NormalTok{, }\StringTok{"DATA_RESPOSTA"}\NormalTok{)}
\CommentTok{#View(head(db_raw))}
\NormalTok{LAI =}\StringTok{ }\NormalTok{db_raw}
\end{Highlighting}
\end{Shaded}

\begin{itemize}
\tightlist
\item
  Respostas e-SIC
\end{itemize}

\begin{Shaded}
\begin{Highlighting}[]
\NormalTok{FILE1 =}\StringTok{ "DATA/relatorio_respostas.xlsx"}
\NormalTok{db1_raw =}\StringTok{ }\NormalTok{readxl}\OperatorTok{::}\KeywordTok{read_excel}\NormalTok{(}\KeywordTok{paste0}\NormalTok{(PATH,FILE1), }\DataTypeTok{sheet =} \StringTok{"DADOS"}\NormalTok{, }\DataTypeTok{col_names =} \OtherTok{TRUE}\NormalTok{); }
\CommentTok{# dim(db1_raw); names(db1_raw)}
\KeywordTok{colnames}\NormalTok{(db1_raw) =}\StringTok{ }\KeywordTok{c}\NormalTok{(}\StringTok{"ID"}\NormalTok{, }\StringTok{"DATA"}\NormalTok{, }\StringTok{"SOLICITACAO"}\NormalTok{, }\StringTok{"DIRETORIA"}\NormalTok{, }\StringTok{"DATA_RESPOSTA"}\NormalTok{)}
\CommentTok{#View(head(db1_raw))}
\NormalTok{LAI1 =}\StringTok{ }\NormalTok{db1_raw}
\end{Highlighting}
\end{Shaded}

\begin{itemize}
\tightlist
\item
  Stopwords
\end{itemize}

\begin{Shaded}
\begin{Highlighting}[]
\NormalTok{FILE2 =}\StringTok{ "DATA/stopwords_PT_FINAL.csv"}
\NormalTok{stopwords_pt =}\StringTok{ }\KeywordTok{read.csv}\NormalTok{(}\KeywordTok{paste0}\NormalTok{(PATH,FILE2), }\DataTypeTok{sep =} \StringTok{';'}\NormalTok{, }\DataTypeTok{header =}\NormalTok{ F, }\DataTypeTok{encoding =} \StringTok{"UTF-8"}\NormalTok{)}
\NormalTok{stopwords_pt =}\StringTok{ }\NormalTok{stopwords_pt[,}\OperatorTok{-}\DecValTok{2}\NormalTok{]; }
\KeywordTok{cat}\NormalTok{(}\KeywordTok{paste0}\NormalTok{(}\StringTok{"O nosso vetor de stopwords contém "}\NormalTok{,}\KeywordTok{length}\NormalTok{(stopwords_pt), }\StringTok{" palavras únicas"}\NormalTok{))}
\end{Highlighting}
\end{Shaded}

\begin{verbatim}
## O nosso vetor de stopwords contém 605 palavras únicas
\end{verbatim}

\begin{Shaded}
\begin{Highlighting}[]
\NormalTok{## dim(stopwords_pt); class(stopwords_pt)}
\NormalTok{stopwords_pt =}\StringTok{ }\KeywordTok{as.character}\NormalTok{(stopwords_pt)}
\NormalTok{stopwords_pt[}\DecValTok{1}\OperatorTok{:}\DecValTok{14}\NormalTok{]}
\end{Highlighting}
\end{Shaded}

\begin{verbatim}
##  [1] ","       "a"       "à"       "acerca"  "adeus"   "agora"   "aí"     
##  [8] "ainda"   "alem"    "além"    "algmas"  "algo"    "algumas" "alguns"
\end{verbatim}

\begin{itemize}
\tightlist
\item
  Dicionário \textgreater{} BASE DE DADOS - REAL PRO TEXTO DO TCC
\end{itemize}

Dicionário de variáveis - PEDIDOS

\begin{Shaded}
\begin{Highlighting}[]
\NormalTok{dicionario =}\StringTok{ "DATA/Dicionario-Dados-Exportacao.txt"}
\NormalTok{dic_pedidos =}\StringTok{ }\KeywordTok{read.delim}\NormalTok{(dicionario, }\DataTypeTok{sep =} \StringTok{"-"}\NormalTok{, }\DataTypeTok{skip =} \DecValTok{3}\NormalTok{, }\DataTypeTok{header =} \OtherTok{FALSE}\NormalTok{, }\DataTypeTok{nrows =} \DecValTok{21}\NormalTok{) }\OperatorTok
\StringTok{  }\KeywordTok{select}\NormalTok{(}\OperatorTok{-}\NormalTok{V1)}
\KeywordTok{colnames}\NormalTok{(dic_pedidos) =}\StringTok{ }\KeywordTok{c}\NormalTok{(}\StringTok{"Nome das variáveis"}\NormalTok{, }\StringTok{"Tipo e descrição da variável"}\NormalTok{)}
\CommentTok{#dimnames(dic_pedidos); View(dic_pedidos)}
\end{Highlighting}
\end{Shaded}

Dicionário de variáveis - RECURSOS

\begin{Shaded}
\begin{Highlighting}[]
\NormalTok{dic_recursos =}\StringTok{ }\KeywordTok{read.delim}\NormalTok{(dicionario, }\DataTypeTok{sep =} \StringTok{"-"}\NormalTok{, }\DataTypeTok{skip =} \DecValTok{30}\NormalTok{, }\DataTypeTok{header =} \OtherTok{FALSE}\NormalTok{, }\DataTypeTok{nrows =} \DecValTok{17}\NormalTok{) }\OperatorTok
\StringTok{  }\KeywordTok{select}\NormalTok{(}\OperatorTok{-}\NormalTok{V1)}
\KeywordTok{colnames}\NormalTok{(dic_recursos) =}\StringTok{ }\KeywordTok{c}\NormalTok{(}\StringTok{"Nome das variáveis"}\NormalTok{, }\StringTok{"Tipo e descrição da variável"}\NormalTok{)}
\CommentTok{#dimnames(dic_recursos); View(dic_recursos)}
\end{Highlighting}
\end{Shaded}

Dicionário de variáveis - SOLICITANTES

\begin{Shaded}
\begin{Highlighting}[]
\NormalTok{dicionario =}\StringTok{ "DATA/Dicionario-Dados-Exportacao.txt"}
\NormalTok{dic_solicitantes =}\StringTok{ }\KeywordTok{read.delim}\NormalTok{(}\DataTypeTok{file =}\NormalTok{ dicionario, }\DataTypeTok{sep =} \StringTok{"-"}\NormalTok{, }\DataTypeTok{skip =} \DecValTok{53}\NormalTok{, }\DataTypeTok{header =} \OtherTok{FALSE}\NormalTok{, }\DataTypeTok{nrows =} \DecValTok{10}\NormalTok{) }\OperatorTok
\StringTok{  }\KeywordTok{select}\NormalTok{(}\OperatorTok{-}\NormalTok{V1)}
\KeywordTok{colnames}\NormalTok{(dic_solicitantes) =}\StringTok{ }\KeywordTok{c}\NormalTok{(}\StringTok{"Nome das variáveis"}\NormalTok{, }\StringTok{"Tipo e descrição da variável"}\NormalTok{)}
\CommentTok{#dimnames(dic_solicitantes); View(dic_solicitantes)}
\end{Highlighting}
\end{Shaded}

\begin{itemize}
\tightlist
\item
  Anexo 02: Tabela - Dicionário de variáveis da tabela de pedidos
\end{itemize}

\begin{Shaded}
\begin{Highlighting}[]
\NormalTok{dic_pedidos }\OperatorTok
\KeywordTok{kable}\NormalTok{(}\StringTok{"latex"}\NormalTok{, }\DataTypeTok{caption =} \StringTok{"Dicionário de variáveis da tabela de pedidos"}\NormalTok{, }\DataTypeTok{booktabs =}\NormalTok{ T) }\OperatorTok
\StringTok{  }\KeywordTok{kable_styling}\NormalTok{(}\DataTypeTok{latex_options =} \KeywordTok{c}\NormalTok{(}\StringTok{"striped"}\NormalTok{, }\StringTok{"hold_position"}\NormalTok{), }\DataTypeTok{full_width =}\NormalTok{ F) }\OperatorTok
\StringTok{  }\CommentTok{#column_spec(1:1, width = "3cm") %>%}
\StringTok{  }\KeywordTok{column_spec}\NormalTok{(}\DecValTok{2}\OperatorTok{:}\DecValTok{2}\NormalTok{, }\DataTypeTok{width =} \StringTok{"15cm"}\NormalTok{) }\OperatorTok
\KeywordTok{landscape}\NormalTok{()}
\end{Highlighting}
\end{Shaded}

\begin{landscape}\rowcolors{2}{gray!6}{white}
\begin{table}[!h]

\caption{\label{tab:unnamed-chunk-4}Dicionário de variáveis da tabela de pedidos}
\centering
\begin{tabular}[t]{l>{\raggedright\arraybackslash}p{15cm}}
\hiderowcolors
\toprule
Nome das variáveis & Tipo e descrição da variável\\
\midrule
\showrowcolors
IdPedido & inteiro: identificador único do pedido (não mostrado no sistema);\\
ProtocoloPedido & texto(17): número do protocolo do pedido;\\
OrgaoSuperiorAssociadoaoDestinatario & texto(250): Quando o órgão for vinculado, este campo traz o nome do seu órgão superior;\\
OrgaoDestinatario & texto(250): nome do órgão destinatário do pedido;\\
Situacao & texto(200): descrição da situação do pedido;\\
\addlinespace
DataRegistro & Data DD/MM/AAAA HH:MM:SS : data de abertura do pedido;\\
ResumoSolicitacao & texto(255): resumo do pedido;\\
DetalhamentoSolicitacao & texto(2048): detalhamento do pedido;\\
PrazoAtendimento & Data DD/MM/AAAA HH:MM:ss : data limite para atendimento ao pedido;\\
FoiProrrogado & texto(3) Sim ou Não : informa se houve prorrogação do prazo do pedido;\\
\addlinespace
FoiReencaminhado & texto(3) Sim ou Não: informa se o pedido foi reencaminhado;\\
FormaResposta & texto(200): tipo de resposta escolhida pelo solicitante na abertura do pedido;\\
OrigemSolicitacao & texto(50): informa se o pedido foi aberto em um Balcão SIC ou pela Internet;\\
IdSolicitante & inteiro: identificador único do solicitante (não mostrado no sistema);\\
CategoriaPedido & texto(200) : categoria do pedido atribuída pel SIC de acordo com o VCGE (Vocabulário COntrolado do GOverno Eletrônico);\\
\addlinespace
SubCategoriaPedido & texto(200) : subcategoria do pedido atribuída pel SIC de acordo com o VCGE (Vocabulário COntrolado do GOverno Eletrônico);\\
NumeroPerguntas & inteiro : número de perguntas feitas no pedido;\\
DataResposta & Data DD/MM/AAAA HH:MM:SS : data da resposta ao pedido (campo em branco para pedidos que ainda estejam na situação Em Tramitação);\\
Resposta & texto(8000): resposta ao pedido;\\
TipoResposta & texto(100) : tipo resposta dada ao pedido (campo em branco para pedidos que ainda estejam na situação Em Tramitação);\\
ClassificacaoTipoResposta & texto(200): subtipo da resposta dada ao pedido (campo em branco para pedidos que ainda estejam na situação Em Tramitação);\\
\bottomrule
\end{tabular}
\end{table}
\rowcolors{2}{white}{white}
\end{landscape}

\begin{itemize}
\tightlist
\item
  Anexo 03: Tabela - Dicionário de variáveis da tabela de recursos
\end{itemize}

\begin{Shaded}
\begin{Highlighting}[]
\NormalTok{dic_recursos }\OperatorTok
\KeywordTok{kable}\NormalTok{(}\StringTok{"latex"}\NormalTok{, }\DataTypeTok{caption =} \StringTok{"Dicionário de variáveis da tabela de recursos"}\NormalTok{, }\DataTypeTok{booktabs =}\NormalTok{ T) }\OperatorTok
\StringTok{  }\KeywordTok{kable_styling}\NormalTok{(}\DataTypeTok{latex_options =} \KeywordTok{c}\NormalTok{(}\StringTok{"striped"}\NormalTok{, }\StringTok{"hold_position"}\NormalTok{), }\DataTypeTok{full_width =}\NormalTok{ F) }\OperatorTok
\StringTok{  }\CommentTok{#column_spec(1:1, width = "3cm") %>%}
\StringTok{  }\KeywordTok{column_spec}\NormalTok{(}\DecValTok{2}\OperatorTok{:}\DecValTok{2}\NormalTok{, }\DataTypeTok{width =} \StringTok{"15cm"}\NormalTok{) }\OperatorTok
\KeywordTok{landscape}\NormalTok{()}
\end{Highlighting}
\end{Shaded}

\begin{landscape}\rowcolors{2}{gray!6}{white}
\begin{table}[!h]

\caption{\label{tab:unnamed-chunk-5}Dicionário de variáveis da tabela de recursos}
\centering
\begin{tabular}[t]{l>{\raggedright\arraybackslash}p{15cm}}
\hiderowcolors
\toprule
Nome das variáveis & Tipo e descrição da variável\\
\midrule
\showrowcolors
IdRecurso & inteiro: identificador único do recurso (não mostrado no sistema);\\
IdRecursoPrecedente & inteiro: identificador único do recurso que precedeu este (não mostrado no sistema e em branco no caso de Recursos de 1ª Instância e Reclamações);\\
DescRecurso & texto(8000): descrição do recurso;\\
IdPedido & inteiro: identificador único do pedido ao qual o recurso pertence (não mostrado no sistema);\\
IdSolicitante & inteiro: identificador único do solicitante (não mostrado no sistema);\\
\addlinespace
ProtocoloPedido & texto(17): número do protocolo do pedido ao qual o recurso pertence;\\
OrgaoSuperiorAssociadoaoDestinatario & texto(250): Quando o órgão for vinculado, este campo traz o nome do seu órgão superior;\\
OrgaoDestinatario & texto(250): nome do órgão destinatário do recurso;\\
Instancia & texto(80): descrição da instância do recurso;\\
Situacao & texto(80): descrição da situação do recurso;\\
\addlinespace
DataRegistro & Data DD/MM/AAAA HH:MM:SS : data de abertura do recurso;\\
PrazoAtendimento & Data DD/MM/AAAA HH:MM:SS : data limite para atendimento ao recurso;\\
OrigemSolicitacao & texto(50): informa se o recurso foi aberto em um Balcão SIC ou pela Internet;\\
TipoRecurso & texto(80): motivo de abertura do recurso;\\
DataResposta & Data DD/MM/AAAA HH:MM:SS : data da resposta ao recurso (campo em branco para recursos que ainda estejam na situação Em Tramitação);\\
\addlinespace
RespostaRecurso & texto(8000): resposta ao recurso;\\
TipoResposta & texto(80): tipo resposta dada ao recurso (campo em branco para recursos que ainda estejam na situação Em Tramitação);\\
\bottomrule
\end{tabular}
\end{table}
\rowcolors{2}{white}{white}
\end{landscape}

\begin{itemize}
\tightlist
\item
  Anexo 04: Tabela - Dicionário de variáveis da tabela de solicitantes
\end{itemize}

\begin{Shaded}
\begin{Highlighting}[]
\NormalTok{dic_solicitantes }\OperatorTok
\KeywordTok{kable}\NormalTok{(}\StringTok{"latex"}\NormalTok{, }\DataTypeTok{caption =} \StringTok{"Dicionário de variáveis da tabela de solicitantes"}\NormalTok{, }\DataTypeTok{booktabs =}\NormalTok{ T) }\OperatorTok
\StringTok{  }\KeywordTok{kable_styling}\NormalTok{(}\DataTypeTok{latex_options =} \KeywordTok{c}\NormalTok{(}\StringTok{"striped"}\NormalTok{, }\StringTok{"hold_position"}\NormalTok{), }\DataTypeTok{full_width =}\NormalTok{ F) }\OperatorTok
\StringTok{  }\CommentTok{#column_spec(1:1, width = "3cm") %>%}
\StringTok{  }\KeywordTok{column_spec}\NormalTok{(}\DecValTok{2}\OperatorTok{:}\DecValTok{2}\NormalTok{, }\DataTypeTok{width =} \StringTok{"15cm"}\NormalTok{) }\OperatorTok
\KeywordTok{landscape}\NormalTok{()}
\end{Highlighting}
\end{Shaded}

\begin{landscape}\rowcolors{2}{gray!6}{white}
\begin{table}[!h]

\caption{\label{tab:unnamed-chunk-6}Dicionário de variáveis da tabela de solicitantes}
\centering
\begin{tabular}[t]{l>{\raggedright\arraybackslash}p{15cm}}
\hiderowcolors
\toprule
Nome das variáveis & Tipo e descrição da variável\\
\midrule
\showrowcolors
IdSolicitante & inteiro: identificador único do solicitante (não mostrado no sistema);\\
TipoDemandante & texto(15): Pessoa Fìsica ou Pessoa Jurídica;\\
DataNascimento & Data DD/MM/AAAA : data de nascimento do solicitante;\\
Sexo & texto(13) : Masculino ou Feminino (em branco para pessoa jurídica);\\
Escolaridade & texto(200): Escolaridade do solicitante (em branco para pessoa jurídica);\\
\addlinespace
Profissao & texto(200): Profissão do solicitante (em branco para pessoa jurídica);\\
TipoPessoaJuridica & texto(200): tipo de Pessoa Jurídica do solicitante (em branco para pessoa física)\\
Pais & texto(200): país de residência do solicitante;\\
UF & texto(2): UF de residência do solicitante;\\
Municipio & texto(200): Município de residência do solicitante;\\
\bottomrule
\end{tabular}
\end{table}
\rowcolors{2}{white}{white}
\end{landscape}


\end{document}
